\documentclass{jsarticle}
\usepackage{amsmath}
\usepackage{bm}
\usepackage{amssymb}

\topmargin = 0mm
\oddsidemargin = 5mm
\textwidth = 152mm
\textheight = 240mm

\title{chapeter 2 problem 解答}

\begin{document}
\maketitle
\renewcommand{\thesubsection}{2.\arabic{subsection})}





\subsection{sasadhfdfkjahfka}

\begin{align}
\nabla f = (\frac{\partial f}{\partial x},\frac{\partial f}{\partial y},\frac{\partial f}{\partial z}) 
\end{align}
の様に書ける。x成分だけ考えると
\begin{align}
\frac{\partial (e^{i(kr - \omega t)})}{\partial x} = ik_{x}f 
\end{align}
y成分、z成分も同様に考えれば、
\begin{align}
\frac{\partial (e^{i(kr - \omega t)})}{\partial y} = ik_{y}f \ ,
\frac{\partial (e^{i(kr - \omega t)})}{\partial z} = ik_{z}f
\end{align}•
したがって、答えは
\begin{align}
\nabla f = (k_{x},k_{y},k_{z})if = if\bf{k}
\end{align}•




\subsection{}
1m離れたところの球面の表面に100Jのエネルギーが入ってくるので、球の表面積でその仕事量を割れば良くて、
\begin{align}
\frac{100}{4\pi} \  \frac{J}{s・m^2}
\end{align}•




\subsection{}
照射強度は単位時間、単位面積あたりのエネルギーなので
\begin{align}
I = \frac{100\times 10^{6}}{\pi(5 \times 10^{-6})} = 1.273 \times 10^{18} \ \frac{J}{m^2・s}
\end{align}•
振幅は以下のように与えられるので
\begin{align}
|E_0|^{2} = \frac{2Z_0}{n} = 2Z_0 I = 2(\frac{\mu_0}{\epsilon_0})^{1/2}I = 3.096\times 10^{10}
\end{align}•



\subsection{}


\subsection{}
\begin{align}
\bm{E} &= E_0(1,be^{i\phi})e^{i(kz-\omega t)} \\
&=E_0(e^{i(kz-\omega t)},be^{i(kz-\omega t+\phi)})
\end{align}•
実数部分だけ見れば実数表示と一致する。



\subsection{}
(a)
\begin{align}
\bm{E} = E_0(\cos(kz-\omega t),\cos(kz-\omega t))
\end{align}

直線偏向になっている($y=x$)
。


(b)
\begin{align}
\bm{E} = E_0(\cos(kz-\omega t),2\cos(kz-\omega t))
\end{align}

直線偏向($y=2x$)

(c)
\begin{align}
\bm{E}& = E_0(\cos(kz-\omega t),-\cos(kz-\omega t+\pi/2))  \nonumber \\
&= E_0(\cos(kz-\omega t),\sin(kz-\omega t)
\end{align}

円偏光になっている。


(d)
\begin{align}
\bm{E} = E_0(\cos(kz-\omega t),\cos(kz-\omega t+\pi/4))
\end{align}

楕円偏光になります。(長軸が45度傾いている)

\subsection{}
(a)
\begin{align}
(1, 1)
\end{align}


(b)
\begin{align}
(1, 2)
\end{align}


(c)
複素表示で書けば、eの共通因子がでてきて、くくれます。
\begin{align}
(1,-i)
\end{align}


(d)
\begin{align}
(\sqrt{2} , 1+i)
\end{align}


\subsection{}
○(1,$\sqrt{3}$)のとき


直線偏向になる(x軸から60°の偏角を持つ)。このベクトルに直交するベクトルでは、
\begin{align}
(\sqrt{3},-1)
\end{align}•


○$(i,-1)$のとき


円偏向になる。このベクトルに直行するベクトルは
\begin{align}
(i,-1)
\end{align}•


○$(1-i,1+i)$のとき
\begin{align}
 \left(
    \begin{array}{c}
      1-i \\
      1+i \\     
    \end{array}
  \right)
&= (1+i) \left(
    \begin{array}{c}
      \frac{1-i}{1+i} \\
      1 \\     
    \end{array}
  \right) \\
&=(1+i) \left(
    \begin{array}{c}
      \frac{-2i}{2} \\
      1 \\     
    \end{array}
  \right) \\
&=(1+i) \left(
    \begin{array}{c}
      -i \\
      1 \\     
    \end{array}
  \right) \\
&= (i-1) \left(
    \begin{array}{c}
      1 \\
      i \\     
    \end{array}
  \right)
\end{align}
$\sigma_+$であることがわかる。つまり、円偏光である。これに直行するベクトルは逆向きの円偏光ベクトルなので,
\begin{align}
\left(
    \begin{array}{c}
      1 \\
     - i \\     
    \end{array}
  \right)
\end{align}

\subsection{}

\subsection{}
入射光を$\left(
    \begin{array}{c}
      A \\
      B \\     
    \end{array}
  \right)$とする。fig2.7の通りの偏光子を設置すると、透過した光はジョーンズ行列を用い書くと、
\begin{align}
\begin{bmatrix}
1 & 0 \\
0 & i
\end{bmatrix}
\frac{1}{2}
\begin{bmatrix}
1 & 1 \\
1 & 1
\end{bmatrix}
\left[
    \begin{array}{c}
      A \\
      B \\     
    \end{array}
  \right]      
&=
\frac{1}{2}
 \begin{bmatrix}
1 & 0 \\
0 & i
 \end{bmatrix}
\begin{bmatrix}
A+B \\
A+B
\end{bmatrix} \\
&=
\frac{1}{2}
 \begin{bmatrix}
A+B \\
(A+B)i
 \end{bmatrix}  
=
\frac{A+B}{2}\begin{bmatrix}
1 \\
i
\end{bmatrix}
\end{align}
となる。ここで、AとBは任意である。AとBの値にかかわらずこの偏光は円偏光になる。A、Bは振幅の大きさを決めるパラメータになっている。

\subsection{}
この問題は固有ベクトルを求める問題になる。ある行列(偏光子)の固有ベクトルは、その固有ベクトルと一致する偏光状態の光を通すと、何も変化しない(振幅だけ変化するが)。なので、固有ベクトルを探すことから始める。
\begin{align}
\begin{bmatrix}
1 & i \\
-i & 1
\end{bmatrix}
\begin{bmatrix}
a  \\
b 
\end{bmatrix}
&=
\lambda
\begin{bmatrix}
a  \\
b 
\end{bmatrix} \\
&=
\begin{bmatrix}
1-\lambda & i \\
-i & 1-\lambda
\end{bmatrix}
\begin{bmatrix}
a  \\
b 
\end{bmatrix} =0
\end{align}•
$\begin{bmatrix}
a  \\
b 
\end{bmatrix}=\begin{bmatrix}
0  \\
0 
\end{bmatrix}$ なので
\begin{align}
\begin{vmatrix}
1-\lambda & i \\
-i & 1-\lambda
\end{vmatrix}
=0 \\
\lambda=0,2
\end{align}•

○$\lambda=0$のとき


$a=-ib$なので固有ベクトルは
\begin{align}
\begin{bmatrix}
1  \\
-i 
\end{bmatrix}
\end{align}•
このとき、$\lambda = 0$なので光を通さない。

○$\lambda=2$のとき


$a=ib$なので固有ベクトルは
\begin{align}
\begin{bmatrix}
1  \\
i 
\end{bmatrix}
\end{align}•
このとき、偏光状態を変えずに光を透過する。



\subsection{}
$\begin{bmatrix}
1  \\
0
\end{bmatrix}$
の光を問題に与えられた順($\frac{1}{2}\begin{bmatrix}
1&1  \\
1 & 1
\end{bmatrix} \  ,   \begin{bmatrix}
0&0  \\
0&1 
\end{bmatrix}$)に置いた偏光子におくと、
\begin{align}
\frac{1}{2}\begin{bmatrix}
1&1  \\
1 & 1
\end{bmatrix}
\begin{bmatrix}
0&0  \\
0&1 
\end{bmatrix}
\begin{bmatrix}
1  \\
0
\end{bmatrix}
&=
\frac{1}{2}
\begin{bmatrix}
0&0  \\
0&1
\end{bmatrix}
\begin{bmatrix}
1  \\
1
\end{bmatrix}  \\
&=
\frac{1}{2}
\begin{bmatrix}
0  \\
1
\end{bmatrix}
\end{align}
この操作で$\begin{bmatrix}
1  \\
0
\end{bmatrix}$
が
$\begin{bmatrix}
0  \\
1
\end{bmatrix}$
に変わったことがわかる。そして、それは偏光面が90°変わったことを意味している。


\subsection{}
$\begin{bmatrix}
1&1 \\
1&1
\end{bmatrix}$
に対する固有ベクトルと固有値を求める。

\begin{align}
\begin{bmatrix}
1&1 \\
1&1
\end{bmatrix}
\begin{bmatrix}
a  \\
b
\end{bmatrix}
=
\lambda
\begin{bmatrix}
a  \\
b
\end{bmatrix} 
\begin{bmatrix}
1&1 \\
1&1
\end{bmatrix} \\
\begin{bmatrix}
1-\lambda&1 \\
1&1-\lambda
\end{bmatrix}
\begin{bmatrix}
a  \\
b
\end{bmatrix} 
=0
\end{align}•
行列式が0になることを使えば、$\lambda=0,2$がわかる。


○$\lambda = 0$のとき


$A=-B$なので固有ベクトルは
\begin{align}
\begin{bmatrix}
1  \\
-1
\end{bmatrix} 
\end{align}


○$ \lambda = 2 $のとき


$A = B$なので固有ベクトル
\begin{align}
\begin{bmatrix}
1  \\
1
\end{bmatrix} 
\end{align}


\subsection{}
○水の場合($n=1.33$)
\begin{align}
\sin^{-1}(\frac{1}{n}) \  \frac{180}{\pi} = 48.75 \ (degree)
\end{align}•

○ダイヤ($n= 2.42$)
\begin{align}
\sin^{-1}(\frac{1}{n})  \ \frac{180}{\pi} = 24.40 \ (degree)
\end{align}

\subsection{}
○水の場合($n = 1.33$)
\begin{align}
\theta_{\rm{Brewater}} = \tan{1.33}^{-1} = 53.06 \ (degree)
\end{align}•

○ダイヤの場合($n = 2.42$)
\begin{align}
\theta_{\rm{Brewater}} = \tan{2.42}^{-1} = 67.54
 \ (degree)
\end{align}


\subsection{}
P偏光、S偏光の反射率は式(2.58),(2.59),(2.60)でわかる様に
\begin{align}
&R_s = \left|r_s \right|^{2}= \left|\frac{\cos{\theta} - \sqrt{n^2-\sin^2{\theta}}}{\cos{\theta} + \sqrt{n^2-\sin^2{\theta}}} \right|^2 \\
&R_p = \left|r_p \right|^{2}= \left|\frac{-n^2\cos{\theta} + \sqrt{n^2-\sin^2{\theta}}}{n^2\cos{\theta} + \sqrt{n^2-\sin^2{\theta}}} \right| ^2
\end{align}•
と書ける。$\theta = \pi/4$の時考える。


○水の場合($n=1.33$)
\begin{align}
R_s &= 0.0523067 \\
R_p &= 0.0027359
\end{align}•


○ダイヤの場合($n = 2.42$)
\begin{align}
R_s &= 0.2829700 \\
R_p &= 0.0800720
\end{align}•

一応pythonのソースコードを貼っておく

*************************************

import numpy as np

a = np.pi/4

n = 

rs = np.cos(a)-np.sqrt(n**2-(np.sin(a))**2)

rs = rs/(np.cos(a)+np.sqrt(n**2-(np.sin(a))**2))

rp =  -n**(2) * np.cos(a)+np.sqrt(n**2-(np.sin(a))**2)

rp = rp / ( n**(2) * np.cos(a)+np.sqrt(n**2-(np.sin(a))**2))

Rs = rs**2
Rp = rp**2

print (Rs,Rp)

**************************************

\subsection{}
内部反射における臨界角が45°とうことがわかるので、そこから以下のことがわかる。
\begin{align}
\pi/4 = \sin^{-1}{1/n} \ \   \therefore \ \frac{1}{n} = \sin\frac{\pi}{4}= \frac{1}{\sqrt{2}}
\end{align}•
従って
\begin{align}
&\theta_{\rm{Brewater}}= \tan^{-1}n=\tan^{-1}(\sqrt{2})=54.73\ (degree) 
\end{align}

\subsection{}
円偏光を作るためには、1回の反射でS偏光とP偏光の位相差が$\pi/4$であればいい(2回反射するので計$\pi/2$の位相がずれる)。したがって、式(2.27)の$\Delta=\pi/4$を満たすような$\theta$を考えればよい。求める$\theta$に関する方程式は式(2.27)を用いて以下の様にかける。
\begin{align}
\tan\pi/8= \frac{\cos\theta\sqrt{\sin^2\theta-1/1.65^2}}{\sin^2\theta}
\end{align}•
$\theta=60°$の時、この式の両辺がだいたい一致していることを示せばいい(解く必要があるとは思わなかったので)。
\begin{align}
&\mbox{(左辺)}=\tan\pi/8=0.4142 \\
&\mbox{(右辺)}=\frac{\cos\pi/3\sqrt{\sin^2\pi/3-1/1.65^2}}{\sin^2\pi/3}=0.4124
\end{align}•
両辺がだいたい一致するので$\theta=60°$の時、すなわち、$A=60°$の時円偏光になることがわかる。

\subsection{}



\end{document}